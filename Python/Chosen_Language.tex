\section{Chosen Language}
% BUG avec l'indentation ici
% Q1
We have chosen Python which is a programming language that we both like. We learned to use it with the course of Artificial Intelligence, we had to implement our projects with Python. This language is intuitive and seems really complete. There is also a great community behind it and this is really helpful when we need to find an answer to our questions.

% Q2
Python supports multiple programming paradigms: object-oriented \cite{martelli2006python} \cite{functional_doc}, functional \cite{functional_doc} \cite{devtome}, meta-programming \cite{mihai} and procedural \cite{martelli2006python} \cite{functional_doc}.
% Q3
It appeared in February 1991 and takes some aspects of its philosophy from ABC \cite{python_why}. Its syntax is inherited in particular from ABC and C languages \cite{wikipediaEN_history_python}. Python's syntax allows us to quickly create things with smaller portions of code (e.g. we don't need to write long lines such as "\textit{public static void main (String args[])}" in Java).
Python's code has been designed to be easily readable \cite{ZenPython} and is not compiled as in Java or C, it is interpreted.
At the opposite of other programming languages, the indentation is really important because the success of the code's interpretation depends on it. Both spaces and tabulations are accepted for the indentation but mixing them can sometimes result in bugs. As many tools don't distinguish tabulations and spaces, it can be hard to debug.
Python uses late binding \cite{python_pep0289} and duck typing is heavily used \cite{wikipediaEN_duck_typing_python}.
As Python is a dynamically typed language, the type is carried by the values, not by the variables. These ones hold references to objects and these references are passed to functions \cite{wikipediaEN_python_syntax_semantics}.
Everything is an object in Python (even classes). Furthermore, classes have a class (metaclass) \cite{python_datamodel}.
\newline

% Q4
Here is an exemple of Fibonacci:
\lstinputlisting[language=Python]{code/example.py}

We can run this program by executing the following command:
\lstinputlisting[language=Bash]{code/example_run.sh}

And this is the output:
\lstinputlisting[language=Bash]{code/example_output.sh}

% Q5
This language can be used in various kind of application domains. Here are some examples \cite{python_applications}:
\begin{enumerate}
    \item{Web and Internet development:
        \begin{itemize}
            \item Frameworks such as Django and Pyramid
            \item Micro-frameworks such as Flask and Bottle
            \item Advanced content management systems such as Plone
        \end{itemize}
    }
    \item{Scientific and numeric:
        \begin{itemize}
            \item SciPy is a collection of packages for mathematics, science, and engineering
            \item Pandas is a data analysis and modeling library
        \end{itemize}
    
    }
    \item Education: we learned programming with Java but Python seems to be more appropriate as it has a simpler syntax for a similar behaviour.
    \item Software development: even big softwares are done in Python. For example, the well known game \emph{Sid Meier's Civilization IV} has been nearly completely implemented in Python.
\end{enumerate}
