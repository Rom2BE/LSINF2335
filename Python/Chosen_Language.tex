\section{Chosen Language}
% Q1
We have chosen Python which is a programming language quite easy to learn and used in several courses to teach the basics of programming (one of us had a introduction to programming during the secondary school with Python). Within our university cursus, we learned it while following the course of Artificial Intelligence because we had to implement our projects in Python. Thousands of packages (they are around 43000) are available in PyPI (the Python Package Index) so you can easily extend your projects with third party modules. Python is very portable and there is also a great community behind it. Therefore, this is really helpful when we need to find an answer to our questions. Another reason is that Google uses Python, their motto is by the way \textit{"Python where we can, C++ where we must"}.
\newline

% Q2
Python is a multi-paradigms language. It supports the following programming paradigms: object-oriented, procedural (and by extension, imperative), functional and last but not least, reflective programming (and by extension, meta-programming) \cite{martelli2006python, python_doc_functional}.

% Q3
It appeared in February 1991 and takes some aspects of its philosophy from ABC \cite{python_doc_why}. Its syntax is inherited in particular from ABC and C languages \cite{wikipediaEN_history_python}. Python's syntax allows us to quickly create things with smaller portions of code (e.g. we don't need to write long lines such as "\textit{public static void main (String args[])}" in Java).

Python's code has been designed to be easily readable such as stated in \textit{"The Zen of Python"} \cite{python_pep0020} and is interpreted. As in Java or C, the code can also be compiled.

At the opposite of other programming languages, the indentation is really important because the success of the code's interpretation depends on it. Both spaces and tabulations are accepted for the indentation but mixing them can sometimes result in bugs. As many tools don't distinguish tabulations and spaces, it can be hard to debug.

As Python is a dynamically typed language, the type is carried by the values, not by the variables. These ones hold references to objects and these references are passed to functions \cite{wikipediaEN_python_syntax_semantics}.

Python uses late binding \cite{python_pep0289} and duck typing is heavily used \cite{wikipediaEN_duck_typing_python}.

Everything is an object in Python, even classes. Furthermore, classes have a class (metaclass) \cite{python_doc_datamodel}.

Python has other interesting characteristics but we wanted to focus on the ones that seemed the most important to us.
\newline

% Q4
Here is an exemple of working code in Python. It is the implementation of the Fibonacci sequence:
\lstinputlisting[language=Python]{code/fibonacci.py}

We can run this program by executing the following command:
\lstinputlisting[language=Bash]{code/fibonacci_run.sh}

And this is the output:
\lstinputlisting[language=Bash]{code/fibonacci_output.sh}

% Q5
This language can be used in various kind of application domains. Here are some examples \cite{python_applications}:
\begin{enumerate}
    \item {Web and Internet development:
        \begin{itemize}
            \item Frameworks such as Django and Pyramid
            \item Micro-frameworks such as Flask and Bottle
            \item Advanced content management systems such as Plone
        \end{itemize}
    }
    \item {Scientific and numeric:
        \begin{itemize}
            \item SciPy is a collection of packages for mathematics, science, and engineering
            \item Pandas is a data analysis and modeling library
        \end{itemize}
    
    }
    \item Education: we learned programming with Java but Python seems to be more appropriate as it has a simpler syntax for a similar behavior.
    \item Software development: even big softwares are written in Python. For example, the well-known game \emph{Sid Meier's Civilization IV} has been nearly completely implemented in Python.
\end{enumerate}
