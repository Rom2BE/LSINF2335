\section{Chosen Language}
% BUG avec l'indentation ici
% Q1
We have chosen Python that is a programming language that we both like. It is intuitive and really complete. There is also a great community behind it and this is really helpful when we need to find an answer to our questions.
% Q2
Python supports multiple programming paradigms: object-oriented \cite{martelli2006python} \cite{functional_doc}, functional \cite{functional_doc} \cite{devtome}, meta-programming \cite{mihai} and procedural \cite{martelli2006python} \cite{functional_doc}.

% Q3
Indentation\\
Data structures : Since Python is a dynamically typed language, Python values, not variables, carry type.\\

% Q4
Here is an exemple of Fibonacci:
\lstinputlisting[language=Python]{code/example.py}

We can run this program by executing the following command:
\lstinputlisting[language=Bash]{code/example_run.sh}

And this is the output:
\lstinputlisting[language=Bash]{code/example_output.sh}

% Q5
This language can be used in various kind of application domains such as \cite{python_applications}:
\begin{enumerate}
    \item{Web and Internet development:
        \begin{itemize}
            \item Frameworks such as Django and Pyramid
            \item Micro-frameworks such as Flask and Bottle
            \item Advanced content management systems such as Plone
        \end{itemize}
    }
    \item{Scientific and numeric:
        \begin{itemize}
            \item SciPy is a collection of packages for mathematics, science, and engineering
            \item Pandas is a data analysis and modeling library
        \end{itemize}
    
    }
    \item Education: we learned programming with Java but Python seems to be more appropriate as it has a simpler syntax for a similar behaviour.
    \item Software Development: even big softwares are done in Python. For example, the well known game \emph{Sid Meier's Civilization IV} has been nearly completely implemented in Python.
\end{enumerate}
