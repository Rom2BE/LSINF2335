\section{Chosen Language}
% BUG avec l'indentation ici
• What language have you chosen (and why)?\\
We have chosen Python that is a programming language that we both like. It is intuitive and really complete. There is also a great community behind it and this is really helpful when we need to find an answer to our questions.\\

• What kind of programming paradigm does this language belong to (functional, procedural, logic, object-oriented, multi-paradigm,...)?\\
Python supports multiple programming paradigms: object-oriented \cite{martelli2006python} \cite{functional_doc}, functional \cite{devtome} \cite{functional_doc}, meta-programming \cite{mihai} and procedural \cite{martelli2006python} \cite{functional_doc}\\

• Give a brief introduction to the core syntax / semantics / concepts of that language.\\
%TODO core syntax / semantics / concepts
Indentation\\
Data structures : Since Python is a dynamically typed language, Python values, not variables, carry type.\\

• Give an illustrative working code example of a typical program written in that language.\\
Here is an exemple of Fibonacci:
\lstinputlisting[language=Python]{example.py}

We can run this program by executing the following command:
\lstinputlisting[language=Bash]{example_run.sh}

And this is the output:
\lstinputlisting[language=Bash]{example_output.sh}

• What kind of typical applications is the language targeted at?\\
This language can be used in various kind of application domains such as \cite{python_applications}:
\begin{enumerate}
    \item{Web and Internet Development : 
        \begin{itemize}
            \item Frameworks such as Django and Pyramid
            \item Micro-frameworks such as Flask and Bottle
            \item Advanced content management systems such as Plone
        \end{itemize}
    }
    \item{Scientific and Numeric :
        \begin{itemize}
            \item SciPy is a collection of packages for mathematics, science, and engineering
            \item Pandas is a data analysis and modeling library
        \end{itemize}
    
    }
    \item{Education : We learned programming with Java but Python seems to be more appropriate as it has a simpler syntax for a similar behaviour.}
    \item{Software Development : Even big softwares are done in Python. For example the well known game \emph{Sid Meier's Civilization IV} has been nearly completely implemented in Python.}
\end{enumerate}
