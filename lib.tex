\documentclass[12pt,a4paper]{article}

% French
\usepackage[utf8x]{inputenc}
%\usepackage[english]{babel}
\usepackage[T1]{fontenc}
\usepackage{lmodern}
\usepackage{url}
\usepackage{cite}

% Math symbols
\usepackage{amsmath}
\usepackage{amssymb}
\usepackage{amsthm}
\usepackage{subfigure} %Allows to have several figures on the same line.
\usepackage{listings} % environnement verbatim améliorée
\usepackage{color}
\usepackage[usenames,dvipsnames,svgnames,table]{xcolor}
\usepackage{tkz-graph}
\usepackage{tikz}
\usetikzlibrary{arrows,matrix,decorations.pathreplacing,positioning,chains,fit,shapes,calc} %Voir 1.3.8
\usepackage{fancyhdr}
\usepackage{hyperref} %Allows to make references (\ref{}), pdf links are now clickable.
\hypersetup{
     colorlinks   = true,
     citecolor    = gray,
     urlcolor     = blue,
     linkcolor    = gray,  
}

% Theorem and definitions
\theoremstyle{definition}
\newtheorem{mydef}{Définition}[subsection]
\newtheorem{mynota}[mydef]{Notation}
\newtheorem{myprop}[mydef]{Propriétés}
\newtheorem{myrem}[mydef]{Remarque}
\newtheorem{myform}[mydef]{Formules}
\newtheorem{mycorr}[mydef]{Corrolaire}
\newtheorem{mytheo}[mydef]{Théorème}
\newtheorem{mylem}[mydef]{Lemme}
\newtheorem{myexem}[mydef]{Exemple}
\newtheorem{myalgo}[mydef]{Algorithme}

\lstset{
	basicstyle=\ttfamily,
	keywordstyle=\bf\color{blue},
	commentstyle=\color{red},
	stringstyle=\color{magenta},
	%numerotation
	numbers=left,
	numberstyle=\tiny\color{gray},
	stepnumber=1,
	numbersep=5pt,
	tabsize=5, % nombre d'espaces d'indentation
	%cadre
	frame=TBlr,
	backgroundcolor=\color{azure},
	rulecolor=\color{blue}	
} 

\lstset{
    language=Python,
    frame=single,
    numbers=left,
    numberstyle=\footnotesize,
    tabsize=2,
    keepspaces=true,
    columns=fullflexible,
    basicstyle=\ttfamily\scriptsize,
    keywordstyle=\color{blue},
    breaklines=true,
    breakatwhitespace=false
}

\lstset{%
    language=bash,
    columns=fullflexible,
    aboveskip=5pt,
    belowskip=10pt,
    basicstyle=\small\ttfamily,
    numbers=left,
    numberstyle=\tiny\color{black!85},
    stepnumber=1, 
    numbersep=13pt,
    backgroundcolor=\color{black!5},
    showspaces=false,
    showstringspaces=false,
    showtabs=false,
    xleftmargin=20pt,
    xrightmargin=10pt,
    framesep=5pt,
    framerule=3pt,
    frame=leftline,
    tabsize=2,
    breaklines=true,
    breakatwhitespace=true,
}